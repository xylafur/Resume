
\documentclass{article}
\usepackage{titlesec}
\usepackage{titling}
\usepackage[papersize={8.5in,11in},margin=0.25in]{geometry}

\hyphenpenalty=10000

\pagenumbering{gobble}
\titleformat{\section}
{\large\bfseries} {} {0em}
{}[\titlerule]

\titleformat{\subsection}
{\normalsize\bfseries}
{$\bullet$}
{0em} {}

\titleformat{\subsubsection}[runin]
{\small\bfseries}
{}
{0em}
{}

\titlespacing{\subsubsection}
{0em}{0.5em}{1em}

\titlespacing{\subsection}
{0em}{0.75em}{0em}

\titlespacing{\section}
{0em}{0.8em}{0.25em}

\renewcommand{\maketitle}{
\begin{center}
{\Large\bfseries\theauthor}
\begin{center}
\begin{tabular}{ r l c r l }
 \textbf{email} & james@jkesley.com & --- & \textbf{phone} & 281 728 6699 \\
 \textbf{github} & xylafur & --- & \textbf{website} & jkesley.com
\end{tabular}
\end{center}
\end{center}
}

\title{R\'esum\'e}
\author{James Kesley Richardson}
\date{}

\begin{document}
\maketitle

\section{Professional Summary}
\begin{flushleft}
\begin{tabular}{p{20cm} p{5.5cm}}

\noindent{Kesley Richardson is a embedded software engineer who is also a committed self learner,
          constantly striving for greater performance, success, and knowledge.  Disciplined in
          creating resilient, maintainable and well documented code.  Working hand in hand with
          software, firmware and hardware engineers at IBM Flashsystems over the last five years
          has given Kesley deep insight and experience over multiple slices of the technology stack.}
    \\
    \\

\end{tabular}
\end{flushleft}


\section{Work Experience}
\begin{flushleft}
\begin{tabular}{p{20cm} p{5.5cm}}

{$ $}\textbf{Embedded Software Engineer, \textit{IBM Flash Systems}} \\
    {$ $}\indent{\textbf{January 2020 - Present}}
\\
\\

\noindent{Wore multiple hats, working across the stack by delivering embedded software solutions,
          crafting scripts for performance and functional testing, assisting in flashcard bring up,
          and contributing quality documentation}
    \\
    \\

    $\ \bullet \ $ Implemented NAND failure recovery mechanisms and created a resilient API to
                   perform Raid5 reconstruction, ensuring data recovery from faulty NAND cells
    \\

    $\ \bullet \ $ Created scripts utilized by the wider team to validate performance and
                   functional behavior of the flashcard
    \\

    $\ \bullet \ $ Assisted in bringing up a new generation of flashcard, implementing device
                   drivers for various essential flashcard functionality
    \\

    $\ \bullet \ $ Designed and crafted mechanisms to manipulate the internal state of the
                   flashcard to allow for quicker filling, leading to quicker performance and
                   functional test results
    \\

    $\ \bullet \ $ Created scripts utilized by the wider team to validate performance and
                   functional behavior of the flashcard, allowing issues to be caught faster and
                   ensuring product quality
    \\

    $\ \bullet \ $ Learned to think across the stack and develop resilient architectures that meet
                   both performance and costs
    \\

\\
\\


{$ $}\textbf{Flashcard Team Intern, \textit{IBM Flash Systems}} \\
    {$ $}\indent{\textbf{January 2017 - December 2019}}
\\
\\

\noindent{Worked on various projects as the Flashcard Team’s intern, developing a rich and diverse
          skill set in Embedded Firmware and Hardware Development, Userspace Programming,
          Scripting, Linux Utilities, System Management and Administration, and Network Infrastructure}
    \\

    $\ \bullet \ $ Created automated build verification methodology to continuously verify the
                   functionality of the latest Hardware and Firmware builds on physical hardware
    \\

    $\ \bullet \ $ Developed Linux user-space utilities to communicate with flashcard over both PCIe
                   and I2C
    \\

    $\ \bullet \ $ Designed, created, oversaw and performed both automated and manual tests
    \\

    $\ \bullet \ $ Developed and delivered various internal full stack web applications
    \\


\end{tabular}
\end{flushleft}


\section{Education}
\noindent{\textbf{University of Houston (2016 - December 2019)}} \\
\indent{\textbf{Major}: Bachelors of Computer Science} \\
\indent{\textbf{Minor}: Computer Engineering Technology and Mathematics} \\
\indent{\textbf{GPA} {3.627}}

\section{Skills}
\begin{tabular}{p{20cm} l}
    $\ \bullet \ $ Three years work experience developing bare metal embedded software solutions in C
    \\

    $\ \bullet \ $ Five years experience developing quality solutions in Python
    \\

    $\ \bullet \ $ Proficiency in Bash scripting and Linux Utilities
    \\

    $\ \bullet \ $ Experience in digesting datasheets and producing resilient device drivers
    \\

    $\ \bullet \ $ Understanding of computer architecture design and implementation
    \\

\end{tabular}


\iffalse

\section{Projects}
\begin{flushleft}
\begin{tabular}{p{15.5cm} p{2.5cm}}

\noindent{\textbf{Autonomous Maze Solving Robot}}

\noindent{$\ \bullet$} Robot equipped with two wheels, line sensor and bump
                       sensors to allow for solving mazes and racing

\noindent{$\ \bullet$} Used hardware generated interrupts from both GPIO pins
                       and system timers to enable real time functionality for
                       the Robot

\noindent{$\ \bullet$} Developed as final project for Real Time Systems and
                       Embedded Programming Course

\noindent{$\ \bullet$} Third Place out of 12 groups in class

    & \textbf{Spring 2019} \\


\noindent{\textbf{Hardware Implemented Automated Gardening System}}

\noindent{$\ \bullet$} Finite State Machine connected to a Soil Moisture sensor
                       and Water Pump, implemented completely with IC Chips

\noindent{$\ \bullet$} Arduino used as ADC and system clock.

\noindent{$\ \bullet$} Developed as final project for Digital Systems Class

\noindent{$\ \bullet$} Awarded Second place out of more than Thirty Groups

    & \textbf{Spring 2019} \\


\noindent{\textbf{Pick Up Drop off World}}

\noindent{$\ \bullet$} Implementation of the Q-Learning Algorithm to optimize
                       solutions to the Pick Up Drop Off World

\noindent{$\ \bullet$} An initially uninformed agent must move all blocks
                       located on Pick Up Locations to Drop off Locations and
                       learn the layout of the world to optimize performance
                       over time.


\noindent{$\ \bullet$} Developed as final project for Artificial Intelligence
    course

    & \textbf{Spring 2019} \\

\end{tabular}
\end{flushleft}

\section{Relevant Courses}
\begin{flushleft}
\begin{tabular}{p{15.5cm} p{2.5cm}}

\noindent{Microprocessor Architecture}
    & \textbf{Summer 2019} \\

\noindent{Real Time Systems and Embedded Programming}
    & \textbf{Spring 2019} \\

\noindent{The Fundamentals of Operating Systems}
    & \textbf{Fall 2017} \\

\noindent{Computer Architecture and Organization}
    & \textbf{Spring 2017} \\


\end{tabular}
\end{flushleft}


\section{Awards}
\subsection{Second Place in Digital Systems Final Project (2019)}
\subsection{University of Houston Deans List (2019)}

\fi

\end{document}

