
\documentclass{article}
\usepackage{titlesec}
\usepackage{titling}
\usepackage[papersize={8.5in,11in},margin=0.25in]{geometry}

\hyphenpenalty=10000

\pagenumbering{gobble}
\titleformat{\section}
{\large\bfseries} {} {0em}
{}[\titlerule]

\titleformat{\subsection}
{\normalsize\bfseries}
{$\bullet$}
{0em} {}

\titleformat{\subsubsection}[runin]
{\small\bfseries}
{}
{0em}
{}

\titlespacing{\subsubsection}
{0em}{0.5em}{1em}

\titlespacing{\subsection}
{0em}{0.75em}{0em}

\titlespacing{\section}
{0em}{0.8em}{0.25em}

\renewcommand{\maketitle}{
\begin{center}
{\Large\bfseries\theauthor}
\begin{center}
\begin{tabular}{ r l c r l }
 \textbf{email} & james@jkesley.com & --- & \textbf{phone} & 281 728 6699 \\
 \textbf{github} & xylafur & --- & \textbf{website} & jkesley.com
\end{tabular}
\end{center}
\end{center}
}

\title{R\'esum\'e}
\author{James Kesley Richardson}
\date{}

\begin{document}
\maketitle

\section{Education}
\noindent{\textbf{University of Houston (2016 - December 2019)}} \\
\indent{\textbf{Major}: Bachelors of Computer Science} \\
\indent{\textbf{Minor}: Computer Engineering Technology and Mathematics} \\
\indent{\textbf{GPA} {3.627}}

\section{Job Experience}
\begin{flushleft}
\begin{tabular}{p{20cm} p{5.5cm}}

{$ $}\textbf{Flashcard Team Intern, \textit{IBM Flash Systems}} \\
    {$ $}\indent{\textbf{January 2017 - Present}}
\\
\\

\noindent{Worked on various projects as the Flashcard Team’s intern, developing a
           rich and diverse skill set in Linux Utilities, Userspace Programming
           and Scripting, System Management and Administration, Network
           Infrastructure, and Embedded Firmware and Hardware Development}
    \\
    \\

    $\ \bullet \ $ Wrote scripts to test, verify and allocate data for embedded
                   processes running on Flashcards
    \\

    $\ \bullet \ $ Creation and enhancement of full stack web utilities
    \\

    $\ \bullet \ $ Developed Linux user-space utilities to communicate directly
                   with flashcard over PCIe and I2C
    \\

    $\ \bullet \ $ Designed, created, oversaw and conducted both automated and
                   manual testing
    \\

    $\ \bullet \ $ Developed build generation and verification processes
    \\

\end{tabular}
\end{flushleft}


\section{Skills}
\begin{tabular}{p{20cm} l}
    $\ \bullet \ $ Proficiency in multiple scripting languages (Python and Bash)
    \\

    $\ \bullet \ $ Two years of experience developing software in C
    \\

    $\ \bullet \ $ Concrete understanding of Computer Architecture
    \\

    $\ \bullet \ $ Fluency in Linux utilities and scripting
    \\

    $\ \bullet \ $ Experience with multiple development tools (Git/Github, Jira, RTC)
    \\

    $\ \bullet \ $ Ability to learn new concepts quickly
    \\

\end{tabular}



\section{Projects}
\begin{flushleft}
\begin{tabular}{p{15.5cm} p{2.5cm}}

\noindent{\textbf{Autonomous Maze Solving Robot}}

\noindent{$\ \bullet$} Robot equipped with two wheels, line sensor and bump
                       sensors to allow for solving mazes and racing

\noindent{$\ \bullet$} Used hardware generated interrupts from both GPIO pins
                       and system timers to enable real time functionality for
                       the Robot

\noindent{$\ \bullet$} Developed as final project for Real Time Systems and
                       Embedded Programming Course

\noindent{$\ \bullet$} Third Place out of 12 groups in class

    & \textbf{Spring 2019} \\


\noindent{\textbf{Hardware Implemented Automated Gardening System}}

\noindent{$\ \bullet$} Finite State Machine connected to a Soil Moisture sensor
                       and Water Pump, implemented completely with IC Chips

\noindent{$\ \bullet$} Arduino used as ADC and system clock.

\noindent{$\ \bullet$} Developed as final project for Digital Systems Class

\noindent{$\ \bullet$} Awarded Second place out of more than Thirty Groups

    & \textbf{Spring 2019} \\


\noindent{\textbf{Pick Up Drop off World}}

\noindent{$\ \bullet$} Implementation of the Q-Learning Algorithm to optimize
                       solutions to the Pick Up Drop Off World

\noindent{$\ \bullet$} An initially uninformed agent must move all blocks
                       located on Pick Up Locations to Drop off Locations and
                       learn the layout of the world to optimize performance
                       over time.


\noindent{$\ \bullet$} Developed as final project for Artificial Intelligence
    course

    & \textbf{Spring 2019} \\

\end{tabular}
\end{flushleft}

\section{Relevant Courses}
\begin{flushleft}
\begin{tabular}{p{15.5cm} p{2.5cm}}

\noindent{Microprocessor Architecture}
    & \textbf{Summer 2019} \\

\noindent{Real Time Systems and Embedded Programming}
    & \textbf{Spring 2019} \\

\noindent{The Fundamentals of Operating Systems}
    & \textbf{Fall 2017} \\

\noindent{Computer Architecture and Organization}
    & \textbf{Spring 2017} \\


\end{tabular}
\end{flushleft}

\iffalse

\section{Awards}
\subsection{Second Place in Digital Systems Final Project (2019)}
\subsection{University of Houston Deans List (2019)}

\fi

\end{document}

